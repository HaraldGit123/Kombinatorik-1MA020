\documentclass[nobib]{tufte-handout}

\title{Föreläsning 4: Fortsättning på inklusion-exklusion $\cdot$ 1MA020}

\author[Vilhelm Agdur]{Vilhelm Agdur\thanks{\href{mailto:vilhelm.agdur@math.uu.se}{\nolinkurl{vilhelm.agdur@math.uu.se}}}}

%\date{15 januari 2023}


%\geometry{showframe} % display margins for debugging page layout

\usepackage{graphicx} % allow embedded images
  \setkeys{Gin}{width=\linewidth,totalheight=\textheight,keepaspectratio}
  \graphicspath{{graphics/}} % set of paths to search for images
\usepackage{amsmath}  % extended mathematics
\usepackage{booktabs} % book-quality tables
\usepackage{units}    % non-stacked fractions and better unit spacing
\usepackage{multicol} % multiple column layout facilities
\usepackage{lipsum}   % filler text
\usepackage{fancyvrb} % extended verbatim environments
  \fvset{fontsize=\normalsize}% default font size for fancy-verbatim environments

\usepackage{color,soul} % Highlights for text

% Standardize command font styles and environments
\newcommand{\doccmd}[1]{\texttt{\textbackslash#1}}% command name -- adds backslash automatically
\newcommand{\docopt}[1]{\ensuremath{\langle}\textrm{\textit{#1}}\ensuremath{\rangle}}% optional command argument
\newcommand{\docarg}[1]{\textrm{\textit{#1}}}% (required) command argument
\newcommand{\docenv}[1]{\textsf{#1}}% environment name
\newcommand{\docpkg}[1]{\texttt{#1}}% package name
\newcommand{\doccls}[1]{\texttt{#1}}% document class name
\newcommand{\docclsopt}[1]{\texttt{#1}}% document class option name
\newenvironment{docspec}{\begin{quote}\noindent}{\end{quote}}% command specification environment

\include{mathcommands.extratex}

\begin{document}

\definecolor{darkgreen}{rgb}{0.0627, 0.4588, 0.1451}

\maketitle% this prints the handout title, author, and date

\begin{abstract}
\noindent
Vi fortsätter studera inklusion-exklusion, och ger fler tillämpningar.
\end{abstract}

\section{Surjektioner}

\begin{definition}
  Låt $A$ och $B$ vara två mängder, och $f: A \to B$ en funktion. Vi definierar \emph{bilden} av $A$ som
  $$f(A) = \left\{b \in B \given \exists a\in A: f(a) = b\right\},$$
  det vill säga alla element i $B$ som träffas av något element i $A$ under $f$.

  Funktionen $f$ är en \emph{surjektion} om $f(A) = B$. Om det finns en surjektion från $A$ till $B$ gäller det att $\abs{A} \geq \abs{B}$.\sidenote[][]{Detta är uppenbart för ändliga mängder $A$ och $B$ -- för oändliga mängder är detta definitionen av ordningen mellan kardinaltal.}
\end{definition}

\begin{definition}
  För $n \geq m \geq 1$ ges \emph{Stirlings partitionstal}, också kallat \emph{Stirlingtalet av andra sorten}, av
  $$\stirlingPart{n}{m} = \frac{1}{m!}\sum_{k=0}^{m}(-1)^k\binom{m}{k}(m-k)^n.$$
\end{definition}

\begin{theorem}\label{theorem_count_surjections}
  Låt $A$ och $B$ vara ändliga mängder med $\abs{A} = n$, $\abs{B} = m$, och $n \geq m$. Antalet surjektioner från $A$ till $B$ ges av
  $$S(n,m) = m!\stirlingPart{n}{m} = \sum_{k=0}^{m} (-1)^k \binom{m}{k}(m-k)^n.$$

  \begin{proof}
    Låt $X$ vara mängden av alla funktioner från $A$ till $B$, och för varje $b \in B$, låt $X_b$ vara mängden av funktioner från $A$ till $B$ som inte träffar $b$. Vi vill, som vanligt, räkna ut $\abs{X \setminus \bigcup_{b\in B} X_b} = \abs{X} - \abs{\bigcup_{b\in B} X_b}$.

    Multiplikationsprincipen ger oss enkelt att $\abs{X} = m^n$ -- varje element i $A$ har $m$ val för var det skall skickas, och vi har $n$ stycken element att göra det valet för.

    Inklusion-exklusion ger oss att
    $$\abs{\bigcup_{b\in B} X_b} = \sum_{I \subseteq B} (-1)^{\abs{I} + 1}\abs{\bigcap_{b \in I} X_b}$$
    och vad vi behöver räkna är antalet funktioner från $A$ till $B$ som undviker att träffa en viss mängd $I$. Ett specialfall ser vi omedelbart -- om $I = B$ måste snittet vara tomt, eftersom elementen i $A$ måste skickas \emph{någonstans}.

    Att räkna dem är relativt enkelt -- en funktion från $A$ till $B$ som inte träffar en viss mängd $I \subset B$ är ju precis en funktion från $A$ till $B \setminus I$, och vi vet att det finns $\abs{B \setminus I}^{\abs{A}} = (m - \abs{I})^n$ sådana. Så vad vi får är att
    $$\abs{\bigcup_{b\in B} X_b} = \sum_{I \subset B, I \neq B} (-1)^{\abs{I} + 1}(m - \abs{I})^n.$$

    Så om vi grupperar den här summan efter storleken på $I$ vet vi att det finns $\binom{m}{k}$ stycken val av $I$ av storlek $k$, så 
    $$\abs{\bigcup_{b\in B} X_b} = \sum_{k=0}^{m-1} (-1)^{k + 1}\binom{m}{k}(m - k)^n$$
    vilket ger oss resultatet, när vi stoppar in detta i $S(n,m) = \abs{X} - \abs{\bigcup_{b\in B} X_b}$.
  \end{proof}
\end{theorem}

\begin{example}
  Antag att en farmor stickat fem filtar åt sina tre barnbarn. På hur många sätt kan hon fördela filtarna, så att varje barn får åtminstone en filt? Eftersom de är handstickade är så klart filtarna \emph{särskiljbara}, så det här är inte ett exempel på de kompositioner vi såg i föreläsning två, utan ett exempel på surjektioner.

  Vår sats säger oss att svaret är $3!\stirlingPart{5}{3} = 150$.
\end{example}

\section{Mängdpartitioner}

Hur många sätt finns det att fördela $n$ objekt i $k$ stycken olika högar? Här har vi en ny variant på räkneproblem -- istället för att vi har objekt som är särskiljbara eller inte så har vi nu osärskiljbara \emph{lådor}.

\begin{theorem}
  Antag att vi har en mängd $X$ med $\abs{X} = n$. Ett sätt att dela upp denna mängd i $k$ osärskiljbara högar\sidenote[][]{Alltså, för den som vill vara formell, ett sätt att skriva $X = A_1 \cup A_2 \cup \ldots \cup A_k$ där $A_i \cap A_j = \emptyset$ för $i \neq j$, där etiketterna på våra $A_i$ inte spelar roll. Eller så kan man se det som en ekvivalensrelation på $X$ med $k$ delar.} kallas för en \emph{mängdpartition} av $X$ i $k$ delar.

  Antalet sådana ges av
  $$\stirlingPart{n}{k}.$$

  \begin{proof}
    Vi bevisar detta genom att räkna antalet surjektioner från $X$ till $[k]$ på två olika sätt.

    Beteckna antalet mängdpartitioner av $X$ i $k$ delar med $m$. Vi kan skapa oss en surjektion från $X$ till $[k]$ genom att först dela upp $X$ i $k$ delar, och sedan ge etiketter från $1$ till $k$ till delarna. Vår surjektion blir då att vi skickar del $i$ till talet $i \in [k]$. Eftersom det finns $k!$ sätt att tilldela etiketterna (etiketteringen är en permutation av längd $k$ från $[k]$) säger oss Multiplikationsprincipen att det måste finnas $k!m$ surjektioner från $X$ till $[k]$.

    Men vi vet också av Teorem \ref{theorem_count_surjections} att antalet surjektioner ges av $k!\stirlingPart{n}{k}$. Alltså måste $m = \stirlingPart{n}{k}$, som önskat.
  \end{proof}
\end{theorem}

\section{Den tolvfaldiga vägen}

När vi talade om kompositioner fördelade vi alltså osärskiljbara objekt i särskiljbara lådor -- och studerade både fallet där lådor fick vara tomma, och när de inte fick det. För surjektioner fördelar vi särskiljbara objekt i särskiljbara lådor, och kräver att varje låda får ett objekt.

Vi börjar ana ett mönster här i hur våra problem kan se ut. Vi kan ha
\begin{enumerate}
  \item Särskiljbara objekt och särskiljbara lådor
  \item Osärskiljbara objekt och särskiljbara lådor
  \item Särskiljbara objekt och osärskiljbara lådor
  \item Osärskiljbara objekt och osärskiljbara lådor
\end{enumerate}
och vi kan ha olika krav på hur objekten fördelas i lådorna
\begin{enumerate}
  \item Generell -- lådor får vara tomma och får innehålla hur många objekt som helst
  \item Surjektiv -- varje låda måste innehålla något objekt
  \item Injektiv -- ingen låda får innehålla mer än ett objekt
\end{enumerate}
så sammanfattningsvis har vi en tabell med tolv stycken tänkbara kombinatorikproblem. Det kommer visa sig att vi i själva verket redan studerat sju av dem.


Under föreläsning två pratade vi om fördelningar av osärskiljbara objekt mellan särskiljbara personer, och gav en formel för deras antal i Proposition 11, men vi gav aldrig ett kort namn åt detta problem. Låt oss kalla en sådan fördelning för en \emph{multi-delmängd} till mängden av personer.

En \emph{multi-mängd} är som en vanlig mängd, fast den får innehålla ett och samma objekt mer än en gång. Så vi betraktar alltså fördelningen av objekt mellan personer som en multi-delmängd genom att tänka att varje person är med i multidelmängden lika många gånger som antalet objekt den fick.

Så låt oss rita denna tabell -- problem vi redan studerat är i svart text, problem vi inte sett innan är i \textcolor{darkgreen}{grön text}. Låt $N$ vara en mängd av $n$ \emph{objekt}, och $X$ vara en mängd av $x$ \emph{lådor}. Vi ser fördelningarna av objekt i lådor som en funktion $f: N \to X$.

\begin{fullwidth}
  \begin{tabularx}{\linewidth}{l|ccc}
      & Generellt $f$ & Injektivt $f$ & Surjektivt $f$\\
      \midrule
    Bägge särskiljbara & \specialcell{Ord ur $X$ av längd $n$\\ $x^n$} & \specialcell{Permutation ur $X$ av längd $n$\\ $\frac{x!}{(x-n)!}$} & \specialcell{Surjektion från $N$ till $X$\\$x!\stirlingPart{n}{x}$} \\
    Osärskiljbara objekt & \specialcell{Multi-delmängd av $X$\\ av storlek $n$\\$\binom{n + x - 1}{n}$} & \specialcell{Delmängd av $X$ av storlek $n$\\$\binom{x}{n}$} & \specialcell{Kompositioner av $n$\\av längd $x$\\$\binom{n - 1}{n - x}$} \\
    Osärskiljbara lådor & \specialcell{\textcolor{darkgreen}{Mängdpartition av $X$}\\ \textcolor{darkgreen}{ i $\leq x$ delar} \\\textcolor{darkgreen}{$\sum_{k=1}^{x} \stirlingPart{n}{k}$}} & \specialcell{\textcolor{darkgreen}{Mängdpartition av $X$}\\ \textcolor{darkgreen}{i $\leq x$ delar av storlek $1$}\\\textcolor{darkgreen}{$1$ om $n \leq x$, $0$ annars}} & \specialcell{Mängdpartition av $N$\\i $x$ delar\\$\stirlingPart{n}{x}$} \\
    Bägge osärskiljbara & \specialcell{\textcolor{darkgreen}{Heltalspartition av $n$ i $\leq x$ delar}\\\textcolor{darkgreen}{$p_x(n + x)$}} & \specialcell{\textcolor{darkgreen}{Sätt att skriva $n$ som}\\\textcolor{darkgreen}{summan av $\leq x$ ettor}\\\textcolor{darkgreen}{$1$ om $n \leq x$, $0$ annars}} & \specialcell{\textcolor{darkgreen}{Heltalspartitioner av $n$}\\ \textcolor{darkgreen}{i $x$ delar} \\\textcolor{darkgreen}{$p_x(n)$}} 
  \end{tabularx}
\end{fullwidth}

\section{Övningar}


%\bibliography{references}
%\bibliographystyle{plainnat}

\end{document}

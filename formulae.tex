\documentclass[nobib]{tufte-handout}

\title{Extramaterial: Formler och räkneregler $\cdot$ 1MA020}

\author[Vilhelm Agdur]{Vilhelm Agdur\thanks{\href{mailto:vilhelm.agdur@math.uu.se}{\nolinkurl{vilhelm.agdur@math.uu.se}}}}

%\date{15 januari 2023}


%\geometry{showframe} % display margins for debugging page layout

\usepackage{graphicx} % allow embedded images
  \setkeys{Gin}{width=\linewidth,totalheight=\textheight,keepaspectratio}
  \graphicspath{{graphics/}} % set of paths to search for images
\usepackage{amsmath}  % extended mathematics
\usepackage{booktabs} % book-quality tables
\usepackage{units}    % non-stacked fractions and better unit spacing
\usepackage{multicol} % multiple column layout facilities
\usepackage{lipsum}   % filler text
\usepackage{fancyvrb} % extended verbatim environments
  \fvset{fontsize=\normalsize}% default font size for fancy-verbatim environments

\usepackage{color,soul} % Highlights for text

% Standardize command font styles and environments
\newcommand{\doccmd}[1]{\texttt{\textbackslash#1}}% command name -- adds backslash automatically
\newcommand{\docopt}[1]{\ensuremath{\langle}\textrm{\textit{#1}}\ensuremath{\rangle}}% optional command argument
\newcommand{\docarg}[1]{\textrm{\textit{#1}}}% (required) command argument
\newcommand{\docenv}[1]{\textsf{#1}}% environment name
\newcommand{\docpkg}[1]{\texttt{#1}}% package name
\newcommand{\doccls}[1]{\texttt{#1}}% document class name
\newcommand{\docclsopt}[1]{\texttt{#1}}% document class option name
\newenvironment{docspec}{\begin{quote}\noindent}{\end{quote}}% command specification environment

\include{mathcommands.extratex}

\setlength{\extrarowheight}{8pt}

\begin{document}

\definecolor{darkgreen}{rgb}{0.0627, 0.4588, 0.1451}

\maketitle% this prints the handout title, author, and date

\begin{abstract}
\noindent
I detta dokument ligger en samling av viktiga resultat och räkneregler, sammanfattade utan bevis.
\end{abstract}


\section{Den tolvfaldiga vägen}

\begin{fullwidth}
  \begin{tabularx}{\linewidth}{l|ccc}
    & Generellt $f$ & Injektivt $f$ & Surjektivt $f$\\
    \midrule
    Bägge särskiljbara & \specialcell{Ord ur $X$ av längd $n$\\ $x^n$} & \specialcell{Permutation ur $X$ av längd $n$\\ $\frac{x!}{(x-n)!}$} & \specialcell{Surjektion från $N$ till $X$\\$x!\stirlingPart{n}{x}$} \\
    Osärskiljbara objekt & \specialcell{Multi-delmängd av $X$\\ av storlek $n$\\$\binom{n + x - 1}{n}$} & \specialcell{Delmängd av $X$ av storlek $n$\\$\binom{x}{n}$} & \specialcell{Kompositioner av $n$\\av längd $x$\\$\binom{n - 1}{n - x}$} \\
    Osärskiljbara lådor & \specialcell{Mängdpartition av $N$\\ i $\leq x$ delar \\$\sum_{k=1}^{x} \stirlingPart{n}{k}$} & \specialcell{Mängdpartition av $N$\\ i $\leq x$ delar av storlek $1$\\$1$ om $n \leq x$, $0$ annars} & \specialcell{Mängdpartition av $N$\\i $x$ delar\\$\stirlingPart{n}{x}$} \\
    Bägge osärskiljbara & \specialcell{Heltalspartition av $n$ i $\leq x$ delar\\$p_x(n + x)$} & \specialcell{Sätt att skriva $n$ som\\summan av $\leq x$ ettor\\$1$ om $n \leq x$, $0$ annars} & \specialcell{Heltalspartitioner av $n$\\ i $x$ delar \\$p_x(n)$} 
  \end{tabularx}
\end{fullwidth}

\section{Räkneregler för genererande funktioner}

\begin{lemma}[Räkneregler för genererande funktioner]
  Antag att vi har en följd $\{a_k\}_{k=0}^\infty$, med genererande funktion $F_a$. Då gäller det att
    \begin{enumerate}
        \item För varje $j \geq 1$ är
        $$\sum_{k = j}^{\infty} a_k x^k = \left(\sum_{k=0}^{\infty}a_k x^k\right) - \left(\sum_{k=0}^{k=j-1} a_kx^k\right) = F_a(x) - \sum_{k=0}^{k=j-1} a_kx^k$$
        \item För alla $m \geq 0$, $l \geq -m$ gäller det att
        $$\sum_{k=m}^{\infty} a_k x^{k + l} = x^l\left(\sum_{k=m}^{\infty} a_k x^{k}\right) = x^l\left(F_a(x) - \sum_{k=0}^{m-1} a_k x^k\right)$$
        \item Det gäller att\sidenote[][]{Denna räkneregel kan förstås generealiseras till att högre potenser av $k$ motsvarar högre derivator -- och om vi istället delar med någon potens av $k$ får vi primitiva funktioner till den genererande funktionen.}
        $$\sum_{k=0}^{\infty} k a_k x^k = \frac{F_a'(x)}{x}.$$
    \end{enumerate}
\end{lemma}

\section{Vanliga genererande funktioner}

\begin{tabularx}{\linewidth}{cc}
  Följd & Genererande funktion\\
  \midrule
  $(1, 0, 0, \ldots)$ & $1$\\
  $(1,1,1,\ldots)$ & $\frac{1}{1-x}$\\
  $a_k = 1$ om $k \leq n$, $0$ annars & $\frac{1 - x^{n+1}}{1 - x}$\\
  Fixt $n$, $a_k = \binom{n}{k}$ & $(1+x)^n$\\
  Fixt $n$, $a_k = \binom{n+k-1}{k}$ & $\frac{1}{(1-x)^n}$\\
  \specialcell{Fibonaccitalen\\$f_0 = f_1 = 1$, $f_{k+1} = f_k + f_{k-1}$ för $k \geq 1$} & $\frac{1}{1 - x - x^2}$\\
  \specialcell{Indikatorfunktion för jämna talen\\$(1,0,1,0,1,0,\ldots)$} & $\frac{1}{1-x^2}$\\
  Catalantalen & $\frac{1 - \sqrt{1 - 4x}}{2x}$
\end{tabularx}

\begin{tabularx}{0.9\linewidth}{cc}
  Följd & Exponentiell genererande funktion\\
  \midrule
  $(1, 0, 0, \ldots)$ & $1$\\
  $(1, 1, 1, \ldots)$ & $e^x$\\
  $(0!, 1!, 2!, 3!, \ldots)$ & $\frac{1}{1-x}$\\
  Fixt $n$, $a_k = \frac{n!}{(n-k)!}$ & $(1 + x)^n$
\end{tabularx}

\section{Sannolikhetsteori}

\begin{lemma}
  Det gäller för alla händelser $A$ och $B$ att
  \begin{itemize}
      \item per definition är $\Prob{A} = \sum_{\omega \in A} \mu(\omega)$,
      \item så $\Prob{A^c} = 1 - \Prob{A}$,
      \item och om $A$ och $B$ har tomt snitt, $A\cap B = \emptyset$, så är $\Prob{A \cup B} = \Prob{A} + \Prob{B}$,
      \item och om de inte nödvändigtvis har tomt snitt har vi att
      $$\Prob{A \cup B} = \Prob{A} + \Prob{B} - \Prob{A \cap B}.$$
      \item $\Prob{A \cap B} = \Prob{A \given B}\Prob{B}$,
      \item och per definition är $A$ och $B$ oberoende precis när $\Prob{A \cap B} = \Prob{A}\Prob{B}$.
  \end{itemize}
\end{lemma}

%\bibliography{references}
%\bibliographystyle{plainnat}

\end{document}

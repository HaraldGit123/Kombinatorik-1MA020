\documentclass[nobib]{tufte-handout}

\title{Föreläsning 6: Fortsättning på genererande funktioner $\cdot$ 1MA020}

\author[Vilhelm Agdur]{Vilhelm Agdur\thanks{\href{mailto:vilhelm.agdur@math.uu.se}{\nolinkurl{vilhelm.agdur@math.uu.se}}}}

%\date{15 januari 2023}


%\geometry{showframe} % display margins for debugging page layout

\usepackage{graphicx} % allow embedded images
  \setkeys{Gin}{width=\linewidth,totalheight=\textheight,keepaspectratio}
  \graphicspath{{graphics/}} % set of paths to search for images
\usepackage{amsmath}  % extended mathematics
\usepackage{booktabs} % book-quality tables
\usepackage{units}    % non-stacked fractions and better unit spacing
\usepackage{multicol} % multiple column layout facilities
\usepackage{lipsum}   % filler text
\usepackage{fancyvrb} % extended verbatim environments
  \fvset{fontsize=\normalsize}% default font size for fancy-verbatim environments

\usepackage{color,soul} % Highlights for text

% Standardize command font styles and environments
\newcommand{\doccmd}[1]{\texttt{\textbackslash#1}}% command name -- adds backslash automatically
\newcommand{\docopt}[1]{\ensuremath{\langle}\textrm{\textit{#1}}\ensuremath{\rangle}}% optional command argument
\newcommand{\docarg}[1]{\textrm{\textit{#1}}}% (required) command argument
\newcommand{\docenv}[1]{\textsf{#1}}% environment name
\newcommand{\docpkg}[1]{\texttt{#1}}% package name
\newcommand{\doccls}[1]{\texttt{#1}}% document class name
\newcommand{\docclsopt}[1]{\texttt{#1}}% document class option name
\newenvironment{docspec}{\begin{quote}\noindent}{\end{quote}}% command specification environment

\include{mathcommands.extratex}

\begin{document}

\definecolor{darkgreen}{rgb}{0.0627, 0.4588, 0.1451}

\maketitle% this prints the handout title, author, and date

\begin{abstract}
\noindent
Vi fortsätter förra föreläsningens diskussion om genererande funktioner, och ger fler exempel och sätt att använda sådana för att lösa kombinatoriska problem.
\end{abstract}

\section{Antal lösningar till en ekvation, med begränsningar}

I slutet på förra föreläsningen studerade vi antalet lösningar till ekvationen
$$x_1 + x_2 + x_3 + x_4 + x_5 = k$$
om vi kräver att alla $x_i$ är ickenegativa heltal. Det var ett första exempel på en mer generell kategori av problem med att räkna lösningar på ekvationer. Låt oss börja med ett lite mer invecklat problem:

\begin{example}
    Hur många lösningar finns det till
    $$x_1 + x_2 + x_3 + x_4 = k$$
    om vi kräver att alla $x_i$ är ickenegativa heltal, men också kräver att $x_2$ är jämnt, att $x_3 \leq 10$, och $x_4$ är udda?

    Låt, för varje $k$, $a_k$ vara antalet sådana lösningar.
    Låt sedan $a_k^1$ vara antalet lösningar till $x_1 = k$ i ickenegativa heltal $x_1$,
    $a_k^2$ vara antalet lösningar till $x_2=k$ i ickenegativa jämna heltal,
    $a_k^3$ vara antalet lösningar till $x_3=k$ i heltal mellan $0$ och $10$,
    och $a_k^4$ vara antalet lösningar till $x_4 = k$ i udda heltal.

    Precis som i förra exemplet studerar vi nu faltningen av dessa fyra följder, och ser att
    $$(a^1 * a^2 * a^3 * a^4)_k = \sum_{\substack{k_1, k_2, k_3, k_4\geq 0\\k_1 + k_2 + k_3 + k_4 = k}} a_{k_1}^1a_{k_2}^2a_{k_3}^3a_{k_4}^4 = a_k.$$

    Så precis som i förra exemplet kan vi få fram genererande funktionen för $a_k$, följden vi faktiskt är intresserade av, genom att plocka fram den genererande funktionen för de enklare följderna.

    Vad genererande funktionen för $a^1$ är vet vi sedan innan -- den är bara en följd av ettor, så dess genererande funktion blir $\frac{1}{1-x}$. Likaledes vet vi sedan innan att följden av $n$ stycken ettor och sedan nollor har genererande funktion $\frac{1 - x^{n+1}}{1-x}$, så genererande funktionen för $a^3$ blir $\frac{1 - x^{11}}{1-x}$.

    Däremot för $a^2$ behöver vi räkna ut något nytt, nämligen den genererande funktionen för följden $1,0,1,0,1,\ldots$, indikatorfunktionen av de jämna talen. Så vi får skriva att
    \begin{align*}
        F_{a^2}(x) &= \sum_{k=0}^{\infty} a^2_k x^k\\
        &= \sum_{\substack{k \geq 0\\k \in 2\Z}} x^k\\
        &= \sum_{i=0}^{\infty} x^{2k}\\
        &= \sum_{i=0}^{\infty} (x^2)^k
    \end{align*}
    och sista raden här kan vi känna igen som genererande funktionen av följden $(1,1,1,1,\ldots)$, \emph{utvärderad i} $x^2$. Så detta är lika med $\frac{1}{1-x^2}$.

    Så vad som återstår är alltså $a^4$, indikatorfunktionen för de udda talen. För att få fram dess genererande funktion kan vi använda vad vi just gjorde för de jämna talen:
    \begin{align*}
        F_{a^4}(x) &= \sum_{k=0}^{\infty} a_k x^k\\
        &= \sum_{\substack{k \geq 1\\k\text{ udda}}} x^k\\
        &= x\sum_{\substack{k \geq 1\\k\text{ udda}}} x^{k-1}\\
        &= x\sum_{\substack{k \geq 0\\k \in 2\Z}} x^k\\
        &= \frac{x}{1 - x^2}.
    \end{align*}

    Så, om vi använder att genererande produkten av en faltning är produkten av de genererande funktionerna, ser vi att
    \begin{align*}
        F_a(x) &= \left(\frac{1}{1-x}\right)\left(\frac{1-x^{11}}{1-x}\right)\left(\frac{1}{1-x^2}\right)\left(\frac{x}{1-x^2}\right)\\
        &= \frac{x(1 - x^{11})}{(1-x)^2(1-x^2)^2}
    \end{align*}
    och ber vi vårt favorit-CAS\sidenote[][]{\emph{Computer Algebra System}, alltså till exempel \emph{WolframAlpha} eller något av dess öppna alternativ.} att Taylorutvidga detta uttryck så får vi att
    $$F_a(x) = x + 2x^2 + 5x^3 + 8x^4 + 14x^5 + 20x^6 + 30x^7 + 40x^8 + \ldots$$
    så att följden av antalet lösningar är
    $$0,1,2,5,8,14,20,30,40,55,70,91,111,138,163,\ldots$$
\end{example}

\section{Övningar}

%\bibliography{references}
%\bibliographystyle{plainnat}

\end{document}

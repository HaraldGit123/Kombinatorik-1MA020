\documentclass[nobib]{tufte-handout}

\title{Extra övningar $\cdot$ 1MA020}

\author[Vilhelm Agdur]{Vilhelm Agdur\thanks{\href{mailto:vilhelm.agdur@math.uu.se}{\nolinkurl{vilhelm.agdur@math.uu.se}}}}

%\date{15 januari 2023}


%\geometry{showframe} % display margins for debugging page layout

\usepackage{graphicx} % allow embedded images
  \setkeys{Gin}{width=\linewidth,totalheight=\textheight,keepaspectratio}
  \graphicspath{{graphics/}} % set of paths to search for images
\usepackage{amsmath}  % extended mathematics
\usepackage{booktabs} % book-quality tables
\usepackage{units}    % non-stacked fractions and better unit spacing
\usepackage{multicol} % multiple column layout facilities
\usepackage{lipsum}   % filler text
\usepackage{fancyvrb} % extended verbatim environments
  \fvset{fontsize=\normalsize}% default font size for fancy-verbatim environments

\usepackage{color,soul} % Highlights for text

% Standardize command font styles and environments
\newcommand{\doccmd}[1]{\texttt{\textbackslash#1}}% command name -- adds backslash automatically
\newcommand{\docopt}[1]{\ensuremath{\langle}\textrm{\textit{#1}}\ensuremath{\rangle}}% optional command argument
\newcommand{\docarg}[1]{\textrm{\textit{#1}}}% (required) command argument
\newcommand{\docenv}[1]{\textsf{#1}}% environment name
\newcommand{\docpkg}[1]{\texttt{#1}}% package name
\newcommand{\doccls}[1]{\texttt{#1}}% document class name
\newcommand{\docclsopt}[1]{\texttt{#1}}% document class option name
\newenvironment{docspec}{\begin{quote}\noindent}{\end{quote}}% command specification environment

\include{mathcommands.extratex}

\begin{document}

\definecolor{darkgreen}{rgb}{0.0627, 0.4588, 0.1451}

\maketitle% this prints the handout title, author, and date

\begin{abstract}
\noindent
När jag skriver föreläsningsanteckningar och övningar till dessa hittar jag ibland extra övningar som hade platsat bland de i slutet av föreläsningsanteckningarna. Jag kan så klart inte lägga till alla övningarna till anteckningarna, för att inte inlämningsuppgifterna skall bli allt för betyngande. Dessa extra övningar hamnar i stället i denna fil.

Uppgifterna kommer inte i någon särskild ordning, utan bara efter ordningen jag hittat dem. Nya uppgifter kommer alltid att läggas till sist i filen, så att numreringen av tidigare uppgifter inte ändras.
\end{abstract}

\begin{xca}
    Ge ett kombinatoriskt bevis\sidenote[][]{Ledtråd: Vänster led räknar antalet sätt att välja en delmängd av storlek $n$ \emph{och} en av storlek $m$. Varför räknar också höger led detta?} för att
    $$\binom{z}{n}\binom{z}{m} = \sum_{k=0}^{n} \binom{n + m - k}{k,\,n-k,\,m-k}\binom{z}{m + n -k},$$
    där $\binom{n + m - k}{k,\,n-k,\,m-k}$ är en multinomialkoefficient.
\end{xca}

\begin{xca}
    Hur många heltal mellan $1$ och $100$ är delbara med $2$, $3$, eller $5$?\sidenote[][]{Använd inklusion-exklusion -- till exempel $30$ är ju delbart med alla tre.}
\end{xca}

\begin{xca}
  Antag att vi har en grupp av $n$ stycken kärlekskranka tonåringar -- varenda en av dem är förälskad i någon tonåring i gruppen\sidenote[][]{Det är tänkbart att någon eller några av dem är narcissist och förälskad i sig själv. Enligt vårt antagande om att varje av dem har precis en person som är kär i dem kommer alltså narcissisten inte ha någon annan som är kär i den, eftersom den redan är kär i sig själv.}, och varje av dem har precis en tonåring i gruppen som är kär i dem.

  Hur många möjliga konfigurationer av förälskelser finns det?
\end{xca}

%\bibliography{references}
%\bibliographystyle{plainnat}

\end{document}
